{
  \setlength{\parindent}{0em}
  \linespread{1}

  \vspace*{-2.1em}

  {
    \centering
    \bffangsong\xiaoer
    毕~~业~~论~~文~~(设~~计)~~~~考~~核 \par
  }
  \addtocontents{toc}{\protect\noindent\fangsong\xiaosi 《浙江大学本科生毕业论文(设计)考核表》...................................................... \vspace*{0.25em} \par}


  \vspace{1.1em}

  {
    \bffangsong\sihao
    一、\; 指导教师对毕业论文(设计)的评语: \par
  }

  \vspace{13em}

  {
    \bffangsong\xiaosi
    \hfill 指导教师(签名) \; \uwave{\hspace{7em}} \hspace*{2em}

    \vspace{1em}

    \hfill \hspace{2em} 年 \hspace{1em} 月 \hspace{1em} 日 \hspace*{2em} \par
  }

  \vspace{0.7em}

  {
    \bffangsong\sihao
    二、 \; 答辩小组对毕业论文(设计)的答辩评语及总评成绩:
  }

  \vspace{14.7em}

  {
    \renewcommand{\arraystretch}{1.5}
    \hfill\begin{table}[h]
    \bfsong\xiaosi
    \centering
    \begin{tabular}{|c|l|l|l|l|l|}
    \hline
    \begin{tabular}[c]{@{}l@{}}成绩\\ 比例\end{tabular} & \begin{tabular}[c]{@{}c@{}}文献综述\\ 占(10\%)\end{tabular} & \begin{tabular}[c]{@{}l@{}}开题报告\\ 占(15\%)\end{tabular} & \begin{tabular}[c]{@{}l@{}}外文翻译\\ 占(5\%)\end{tabular} & \begin{tabular}[c]{@{}l@{}}毕业论文(设计)\\ 质量及答辩\\ 占(70\%)\end{tabular} & 总评成绩 \\ \hline
    分值                                              &                                                        &                                                        &                                                       &                                                                    &      \\ \hline
    \end{tabular}
    \end{table}

    % \hfill \begin{tabular}{|c|m{4.1em}|m{4.1em}|m{4.1em}|m{9.1em}|c|}
    %   \hline
    %   成绩比例 & {\centering 文献综述 \\ 占(10\%)} & {\centering 开题报告 \\ 占(15\%)} & {\centering 外文翻译 \\ 占(5\%) } & {\centering 毕业论文(设计) \\ 质量及答辩 \\ 占(70\%)} & 总评成绩 \\
    %   \hline
    %   分值 & & & & & \\
    %   \hline
    % \end{tabular} \par

  }

  \vspace{2em}

  {
    \bffangsong\xiaosi
    \hfill 答辩小组负责人(签名) \; \uwave{\hspace{7em}} \hspace*{2em}

    \vspace{1em}

    \hfill \hspace{2em} 年 \hspace{1em} 月 \hspace{1em} 日 \hspace*{2em} \par
  }
}
