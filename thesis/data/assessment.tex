{
  \setlength{\parindent}{0em}
  \linespread{1}

  \vspace*{-2.1em}

  {
    \centering
    \stfangsong\xiaoer\bfseries
    毕~~业~~论~~文~~(设~~计)~~~~考~~核 \par
  }
  \addtocontents{toc}{\protect\noindent\stfangsong\xiaosi 《浙江大学本科生毕业论文(设计)考核表》.............................................................................\vspace*{0.25em} \par}


  \vspace{1.1em}

  {
    \stfangsong\sihao\bfseries
    一、\; 指导教师对毕业论文(设计)的评语: \par
  }

  \vspace{13em}

  {
    \stfangsong\xiaosi\bfseries
    \hfill 指导教师(签名) \; \uwave{\hspace{5em}}

    \vspace{0.1em}

    \hfill \hspace{2em} 年 \hspace{1em} 月 \hspace{1em} 日 \par
  }

  \vspace{0.7em}

  {
    \stfangsong\sihao\bfseries
    二、 \; 答辩小组对毕业论文(设计)的答辩评语及总评成绩:
  }

  \vspace{14.7em}

  {
    \renewcommand{\arraystretch}{1.5}
    \songti\xiaosi\bfseries
    \hfill \begin{tabular}{|c|m{4.1em}|m{4.1em}|m{4.1em}|m{9.1em}|c|}
      \hline
      成绩比例 & {\centering 文献综述 \\ 占(10\%)} & {\centering 开题报告 \\ 占(15\%)} & {\centering 外文翻译 \\ 占(5\%) } & {\centering 毕业论文(设计) \\ 质量及答辩占(70\%)} & 总评成绩 \\
      \hline
      分值 & & & & & \\
      \hline
    \end{tabular} \par
  }

  \vspace{2em}

  {
    \stfangsong\xiaosi\bfseries
    \hfill 答辩小组负责人(签名) \; \uwave{\hspace{5em}}

    \vspace{0.1em}

    \hfill \hspace{2em} 年 \hspace{1em} 月 \hspace{1em} 日 \par
  }
}
